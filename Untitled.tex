\documentclass[12pt]{article}

\usepackage[utf8]{inputenc}
\usepackage{tex/codeblocks}
\usepackage{booktabs}
\usepackage{array}

\renewcommand{\arraystretch}{1.25}

\begin{document}

\section{The C++ Build Process}

\noindent
Recall that C++ is a compiled language.
This means its source code must be translated into machine-readable instructions, typically resulting in an executable file.
The process by which this translation occurs is known as the \textbf{build process}, which consists of three major stages: preprocessing, compilation, and linking:

\[\texttt{.cpp} \rightarrow \boxed{\textrm{Preprocessor}} \rightarrow \boxed{\textrm{Compiler}} \rightarrow \boxed{\textrm{Linker}} \rightarrow \texttt{exe}\]

\subsection{The Preprocessor}

\noindent
The preprocessor's job is to edit the source code text according to a set of \textbf{preprocessor directives}.
These directives are special instructions that guide the preprocessor before actual compilation begins.
All preprocessor directives begin with the \inlinecxx{#} symbol and are processed line-by-line.

\subsubsection{Text Replacement \inlinecxx{\#define}}

\noindent
The \inlinecxx{#define} directive is used to define macros.
In C/C++, a \textbf{macro} is a symbol that the preprocessor replaces with a value, expression, or code snippet.
\inlinecxx{#define} instructs the preprocessor to replace every occurrence of a defined symbol with a corresponding replacement string.
This allows you to define constants without using variables:

\begin{lstlisting}[style=cxx]
#include <iostream>

#define PI 3.14159 // replace 'PI' with '3.14159'

int main()
{
    std::cout << PI << std::endl;
}
\end{lstlisting}

\noindent
In the code above, the preprocessor replaces every instanc eof \inlinecxx{PI} with \inlinecxx{3.14159}.
It is important to note that the preprocessor has no understanding of C++ semantics--it merely performs straightforward text substitution.
This means it does not perform any type checking or validation, which can lead to unexpected behavior if not used carefully.

\vspace{1em}
\noindent
The \inlinecxx{#define} directive can also be used to create function-like macros that accept arguments:

\begin{lstlisting}[style=cxx]
#include <iostream>

#define SQUARE(x) x * x
#define MAX(a, b) a > b ? a : b

int main()
{
    std::cout << SQUARE(5) << std::endl;
    std::cout << MAX(2, 3) << std::endl;
}
\end{lstlisting}

\noindent
These function like macros are extremely prone to unexpected behavior.
Consider the following example:

\begin{lstlisting}[style=cxx]
#include <iostream>

#define SQUARE(x) x * x

int main()
{
    int a { 2 }, b { 4 };
    std::cout << SQUARE(a + b) << std::endl;
}
\end{lstlisting}

\begin{terminal}
$ ./a.out
14
\end{terminal}

\begin{explanation}
\end{explanation}

\subsection{Conditional Compilation \inlinecxx{\#if}, \inlinecxx{\#ifdef}}

\noindent
The preprocessor can also be used to conditionally pass code to the compiler.
The \inlinecxx{#if} directive passes the following block of code to the compiler if the given condition is true:

\begin{lstlisting}[style=cxx]
#define CONTROL 1

#if (CONTROL == 1)
    // do something...
#endif
\end{lstlisting}

\noindent
The condition passed must be made up of preprocesor macros so that the preprocesor can understand it.
The block must be terminated with the \inlinecxx{#endif} directive.
\inlinecxx{#elif} and \inlinecxx{#else} can be used to create preprocessor branching behavior:

\begin{lstlisting}[style=cxx]
#define CONTROL 1

#if (CONTROL == 0)
    // do something...
#elif (CONTROL == 1)
    // do something else...
#else
    // do something else...
#endif
\end{lstlisting}

\noindent
\inlinecxx{#ifdef} can be used to conditionally evaluate if a macro is defined:

\begin{lstlisting}[style=cxx]
#define MACRO

#ifdef MACRO
    // do something...
#endif
\end{lstlisting}

\noindent
Similarly, \inlinecxx{#ifndef} can be used to execute code if a macro is \textbf{not} defined:

\begin{lstlisting}[style=cxx]
#ifndef MACRO
    // do something...
#endif
\end{lstlisting}

\pagebreak

\begin{table}[ht]
    \centering
    \renewcommand{\arraystretch}{1.25}
    \begin{tabular}{>{\raggedright\arraybackslash}p{0.2\textwidth} >{\raggedright\arraybackslash}p{0.6\textwidth} >{\raggedright\arraybackslash}p{0.15\textwidth}}
        \toprule[1.5pt]
        \textbf{Directive} & \textbf{Description} & \textbf{Section} \\
        \midrule
        {\inlinecxx{#define}}   & Defines a macro, which can be an object-like macro or a function-like macro. & 0 \\
        {\inlinecxx{#undef}}    & Undefines a previously defined macro. &  \\
        {\inlinecxx{#include}}  & Includes the contents of a specified file. &  \\
        {\inlinecxx{#if}}       & Begins a conditional block; code within is compiled only if the condition is true. &  \\
        {\inlinecxx{#elif}}     & Specifies an alternate condition within a {\inlinecxx{#if}} block. &  \\
        {\inlinecxx{#else}}     & Specifies the code to be compiled if none of the preceeding {\inlinecxx{#if}} or {\inlinecxx{\#elif}} checks are true. &  \\
        {\inlinecxx{#ifdef}}    & Checks if a macro is defined; code within is compiled only if the macro is defined. &  \\
        {\inlinecxx{#ifndef}}   & Checks if a macro is not defined; code within is compiled only if the macro is not defined. &  \\
        {\inlinecxx{#elifdef}}  &  &  \\
        {\inlinecxx{#elifndef}} &  &  \\
        {\inlinecxx{#endif}}    & Ends a conditional compilation block. &  \\
        {\inlinecxx{#error}}    & Generates a compilation error with a specified message. &  \\
        {\inlinecxx{#warning}}  &  &  \\
        {\inlinecxx{#line}}     &  &  \\
        {\inlinecxx{#pragma}}   &  &  \\
        \bottomrule[1.5pt]
    \end{tabular}
\end{table}

\end{document}
