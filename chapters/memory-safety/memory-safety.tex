\documentclass[12pt]{article}

\usepackage{/Users/ryanbaker/cxxsaturdays/tex/preamble}

\begin{document}

\tableofcontents

\pagebreak

\section{Memory Safety}

\noindent
Memory safety refers to the concept of ensuring that a program does not access memory in an unintended or unsafe manner.
This can involve issues such as out-of-bounds access, null pointer dereferencing, or memory leaks, all of which can lead to undefined behavior, crashes, or security vulnerabilities.
In C++, memory safety is a critical concern due to the language's low-level features and manual memory management capabilities, which allow developers more control but also more responsibility.

\subsection{Common Memory Safety Issues}

\noindent
Before diving into strategies for improving memory safety, let's examine some of the most common issues developers face in C++ programs.

\subsubsection{Buffer Overflows}

\noindent
Buffer overflows occur when a program writes past the end of an array (a buffer), corrupting adjacent memory.
They’re a common source of security vulnerabilities, as attackers can exploit them to inject or execute arbitrary code.

\begin{cxx}{overflow.cpp}
#include <iostream>

int main()
{
    char buffer[10];
    strcpy(buffer, "This is too long for the buffer");
}
\end{cxx}

\subsubsection{\inlinecxx{nullptr} Dereferencing}

\noindent
Dereferencing a \inlinecxx{NULL} pointer causes undefined behavior and can lead to program crashes:

\begin{cxx}{}
int* ptr { nullptr };
*ptr = 10;
\end{cxx}

\subsubsection{Dangling Pointers}

\noindent
A dangling pointer is a pointer that references a memory location after the object that it points to has been deallocated.
Dereferencing a pointer leads to undefined behavior:

\begin{cxx}{}
int* ptr { new int };
delete ptr;
*ptr = 10;
\end{cxx}

\subsubsection{Memory Leaks}

\noindent
Memory leaks occur when a program allocates memory but fails to deallocate it, causing the program to consume ever-increasing amounts of memory.
Usually this happens when a call to \inlinecxx{delete} does not follow a call to \inlinecxx{new}.

\subsection{Smart Pointers}

\noindent
\textbf{Smart pointers} are wrapper classes that manage the lifetime of dynamically allocated objects.
Unlike raw pointers, which require explicit memory management (using \inlinecxx{new} and \inlinecxx{delete}), smart pointers automatically handle memory cleanup.
Smart pointers are a critical component of RAII (Resource Acquisition Is Initialization), a C++ programming paradigm that ensures resources are properly cleaned up when an object goes out of scope.

\vspace{1em}
\noindent
C++ offers several types of smart pointers, each designed for different use cases.
The most commonly used smart pointers are \inlinecxx{std::unique_ptr}, \inlinecxx{std::shared_ptr}, and \inlinecxx{std::weak_ptr}, all of which are defined in the \inlinecxx{<memory>} header.

\subsubsection{Unique Pointer}

\noindent
Unique pointers are smart pointers that own an object exclusively.
They help to ensure that the resource managed is freed when the pointer goes out of scope.
A unique pointer cannot be copied.
Here is a simplified implementation of the unique pointer class:

\begin{cxx}{uniqueptr.h}
template<class T>
class UniquePtr
{
private:
    T* ptr;

public:
    unique_ptr(): ptr(nullptr) {};
    unique_ptr(T* ptr): ptr(ptr) {}

    unique_ptr(const unique_ptr&) = delete;
    unique_ptr& operator=(const unique_ptr&) = delete;

    ~unique_ptr()
    {
        delete ptr;
    }

    T* operator->() { return ptr; }
};
\end{cxx}

\noindent
Here is an example of using \inlinecxx{std::unique_ptr}:

\begin{cxx}{}
#include <iostream>
#include <memory>

int main()
{
	std::unique_ptr<int> ptr = std::make_unique<int>(10);
	std::cout << *ptr << std::endl;
}
\end{cxx}

\subsubsection{Shared Pointer}

\noindent
A shared pointer allows multiple pointers to share ownership of a resource.
The resource is only freed when the last reference is destroyed.
This is managed by a reference count, which tracks how many shared pointers are sharing the resource:

\begin{cxx}{shared.cpp}
#include <iostream>
#include <memory>

class MyClass
{
public:
	MyClass() { std::cout << "constructed" << std::endl; }
	~MyClass() { std::cout << "destroyed" << std::endl; }
};

int main()
{
	std::shared_ptr<MyClass> ptr { nullptr };
	{
		std::shared_ptr<MyClass> myobj { std::make_shared<MyClass>() };
		ptr = myobj;
	}
	std::cout << "left scope" << std::endl;
}
\end{cxx}

\subsubsection{Weak Pointer}

\noindent
A \inlinecxx{std::weak_ptr} is used to ``observe'' an object that is managed by a shared pointer without affecting its reference count.
Weak pointers can be constructed from shared pointers:

\begin{cxx}{weak.cpp}
#include <iostream>
#include <memory>

int main()
{
    std::weak_ptr<int> weak;
	std::shared_ptr<int> s;

    {
        std::shared_ptr<int> shared = std::make_shared<int>(42);
        weak = shared;

        std::cout << "Inside block: weak expired? " 
                  << std::boolalpha << weak.expired() << "\n";
    }

    std::cout << "Outside block: weak expired? " 
              << weak.expired() << "\n";

    if (auto locked = weak.lock())
	{
        std::cout << "Value: " << *locked << "\n";
    } else
	{
        std::cout << "Object no longer exists.\n";
    }
}
\end{cxx}

\end{document}
