\documentclass[12pt]{article}

\usepackage{/Users/ryanbaker/cxxsaturdays/tex/preamble}

\begin{document}

\tableofcontents

\pagebreak

\section{Pointers}

\noindent
A \textbf{pointer} is a variable that represents a memory address, often the address of another variable.
Pointers ``point'' to locations in memory and allow indirect access to the values stored there.

\subsection{Declaring and Defining Pointers}

\noindent
To declare a pointer, use an asterisk \inlinecxx{*} and specify the type of data it points to:

\begin{cxx}{}
int* ptr; // declares a pointer to an int
\end{cxx}

\noindent
To make a pointer ``point'' to a variable, use the address-of operator \inlinecxx{&}:

\begin{cxx}{}
int* ptr;
int x {};
ptr = &x; // 'point' ptr to the address of x
\end{cxx}

\subsection{Using Pointers}

\noindent
To access the value pointed to, we use the dereference operator \inlinecxx{*}:

\begin{cxx}{dereference.cpp}
#include <iostream>

int main()
{
	int x { 42 };
	int* ptr { &x };
	std::cout << "x = " << *ptr << std::endl;
}
\end{cxx}

\begin{terminal}
$ clang++ -std=c++23 dereference.cpp
$ ./a.out
x = 42
\end{terminal}

\noindent
Since pointers provide direct memory access, they can also be used to modify variables indirectly:

\begin{cxx}{increment.cpp}
#include <iostream>

int main()
{
	int x { 42 };
	int* ptr { &x };
	(*ptr)++;
	std::cout << "x = " << x << std::endl;
}
\end{cxx}

\begin{terminal}
$ clang++ -std=c++23 increment.cpp
$ ./a.out
x = 43
\end{terminal}

\end{document}
