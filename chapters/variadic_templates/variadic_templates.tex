\documentclass[12pt]{article}

\usepackage{../../tex/preamble}

\begin{document}

\section{Variadic Templates}

\noindent
Since C++11, templates can accept a variable number of template parameters.
This feature, known as \textbf{variadic templates}, allows you to pass an arbitrary number of arguments of arbitrary types to a template.
It is especially useful in scenarios where you need flexibility in the number and types of arguments.
For example, you can use variadic templates to create a function that prints an arbitrary set of objects:

\begin{lstlisting}[style=cxx]
#include <iostream>

void print() {}

template<class T, class... Types>
void print(T first, Types... args)
{
    std::cout << first << std::endl;
    print(args...);
}

int main()
{
    print(5, 6, 7.0, "Hello, World!");
}
\end{lstlisting}

\noindent
If one or more arguments are passed to \inlinecxx{print()}, the function template is used.
\inlinecxx{print()} calls itself recursively for the remaining arguments.
The remianing arguments names \inlinecxx{args} are called a \textbf{parameter pack}.

\end{document}
