\documentclass[12pt]{article}

\usepackage{../../tex/preamble}

\begin{document}

\section{Variable Templates}

\noindent
One notable feature introduced in C++14 is the \textbf{variable template}. Variable templates allow us to define parameterized constants, making it easy to work with constants of different types. Here's a simple example of how we can define $\pi$ using a variable template:

\begin{lstlisting}[style=cxx]
template<class T>
const T pi = T(3.14159265358979323);
\end{lstlisting}

\noindent
In this code, \inlinecxx{pi} is a variable template that can be instantiated with various types. The variable template allows $\pi$ to be used with different data types, such as \inlinecxx{double} or \inlinecxx{int}, without needing to write separate definitions for each type.

\begin{lstlisting}[style=cxx]
#include <iostream>

template<class T> const T pi = T(3.14159265358979323);

int main()
{
    using namespace std;
    cout << "math:" << pi<double> << endl; // 3.14159
    cout << "engineering:" << pi<int> << endl; // 3
}
\end{lstlisting}

\end{document}
