\section{Functions}

\noindent
A \textbf{function} is a reusable block of code designed to perform a specific task.
Instead of writing the same code repeatedly, functions let us define it once and reuse it as needed.
This approach keeps our codebase cleaner and easier to maintain.
Proper use of functions also make programs more readable by clearly expressing the programmer's intent.
Consider the following program:

\begin{lstlisting}[style=cxx]
#include <iostream>

int main()
{
    const double F { 98.6 };
    std::cout << ((F - 32.0) * 5.0 / 9.0) << std::endl;
}
\end{lstlisting}

\noindent
If you're quick to recognize scientific formulas, you'll notice that this program performs a Farenheit-to-Celsius conversion.
However, there are a few minor issues with it.
First, to the non-formulae recognizing masses, the program's purpose is obscure.
Second, it does not adapt well to maintenance or extension.
Consider how the program looks when we perform multiple conversions:

\begin{lstlisting}[style=cxx]
#include <iostream>

int main()
{
    double F { 98.6 };
    std::cout << ((F - 32.0) * 5.0 / 9.0) << std::endl;

    F = 32.0; // we had to remove 'const' to reassign to 'F'
    std::cout << ((F - 32.0) * 5.0 / 9.0) << std::endl;

    F = 3038.0; // melting point of human bone
    std::cout << ((F - 32.0) * 5.0 / 9.0) << std::endl;

    F = -166.0 // Jupiter's surface temperature
    std::cout << ((F - 32.0) * 5.0 / 9.0) << std::endl;
}
\end{lstlisting}

\noindent
The above program is needlessly lengthy and suffers from obscurity.
It is also rather difficult to maintain in the event that we need to change the exact nature of our computation.
But all is not lost.
Notice that the logic on lines 5-6, 8-9, 11-12, and 14-15 is identical.
This makes it an ideal candidate to be extracted into a function.

\subsection{Declaring and Defining Functions}

\noindent
Building upon the motivation described above, we will design a function called \inlinecxx{F_to_C()} to perform a Farenheit-to-Celsius conversion.

\begin{advice}
Naming things is one of the great unsolved problems of computer science.
It is always smart to name your functions according to their job.
An ideal function name makes the function's purpose clear to the reader.
\end{advice}

\noindent
The first thing to consider when designing a function in C/C++ is the data type that it will {return}, its \textbf{return type}.
For example, a function that calculates the sum of two integers is likely to return an integer.
Because \inlinecxx{F_to_C()} returns a real number value, its return type will be \inlinecxx{double}.
\begin{lstlisting}[style=cxx]
// write the function's return type before its name
double F_to_C();
\end{lstlisting}

\noindent
\inlinecxx{F_to_C()} needs to accept as input a real number Farenheit value to convert.
In C/C++, function input arguments are specified as a comma separated list of types and names.

\begin{lstlisting}[style=cxx]
double F_to_C(double F);
\end{lstlisting}

\noindent
The above is a valid \textit{function declaration}. Recall that variable initialization can be split into declaration and definition. To \textit{define} a function is to write its body:

\begin{lstlisting}[style=cxx]
void print_divmod(int p, int q)
{
	int d { p / q };
	int r { p % q };
    std::cout << p << " / " << q << " = " << d;
    std::cout << " remainder " << r << '\n';
}
\end{lstlisting}

\subsection{Calling Functions}

\noindent
To call a function, we use the function call operator \inlinecxx{()}:

\begin{lstlisting}[style=cxx]
// ... print_divmod() defined above

int main()
{
    print_divmod(55, 10);
    print_divmod(40, 17);
    print_divmod(144, 1000);
}
\end{lstlisting}

\noindent
If the function has a return type, we use the \inlinecxx{return} keyword to return a piece of data back to the caller. The output of the function can then be assigned to a variable:

\begin{lstlisting}[style=cxx]
#include <iostream>

double mean(double a, double b) { return (a + b) / 2.0; }

int main()
{
    double avg { mean(10, 20) };
    std::cout << avg << std::endl;
}
\end{lstlisting}

\subsection{Function Arguments}

\noindent
By default in C++, arguments are passed as \textit{copies} into functions.
This means that the memory occupied by the variable passed is \textbf{not} the same as the memory occupied by the variable recieved.
Instead, a copy of the variable is made and passed to the function.
We can demonstrate this fact as follows:

\begin{lstlisting}[style=cxx]
#include <iostream>

void foo(int x)
{
    x = 0; // reassign 'x' to 0
}

i
nt main()
{
    int x { 10 };
    foo(x); call foo() with x
    std::cout << x << std::endl;
}
\end{lstlisting}

\begin{terminal}
./a.out
10
\end{terminal}

\noindent
In the above program, we pass a variable into a fucntion that appears to modify it.
However, the original variable remains unchanged.
This is because the variable was \textit{passed by copy} to the function.

\subsection{Function Signatures}

\noindent
A foundational concept relating to functions in C/C++ is a \textit{function signature}.
Every function has a signature.
A funciton's signature is made up of its name and argument types (and its surrounding \inlinecxx{namespace}), and it serves as a unique identifier by which the compiler can refer to the function.
Because the function argument types are a part of its signature, we can define two functions with the same name but different arguments:

\begin{lstlisting}[style=cxx]
int add(int x, int y);
int add(int x, int y, int z);
double add(double x, double y);
\end{lstlisting}

\noindent
The compiler will not get confused about which function we are refering to because it can deduce pased on the arguments passed:

\begin{lstlisting}[style=cxx]
#include <iostream>

int add(int x, int y)          { return x + y; }
int add(int x, int y, int z)   { return x + y + z; }
double add(double x, double y) { return x + y; }

int main()
{
    int i { 10 };
    double d { 7 };

    std::cout << add(i, i) << std::endl;
    std::cout << add(i, i, i) << std::endl;
    std::cout << add(d, d) << std::endl;
}
\end{lstlisting}

\noindent
This is known as function overloading. The clumsy nature of the above code provides basic motivation for \textit{templates}.
