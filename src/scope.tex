\section{Scope}

\noindent
A variable's \textbf{scope} is the region of the program where its declaration is visible.
Every variable declaration occurs within some scope which determines that variable's lifetime and visibility.
Every program has a global scope, which encompasses the entire program.
Narrower scopes are typically denoted with curly braces \inlinecxx{\{\}} in C/C++:

\begin{lstlisting}[style=cxx]
void foo()                          //              //
{                                   // foo()'s      //
    // do foo things...             // scope        //
}                                   //              //
                                                    //
int main()                          //              //
{                                   //              // global
    {                //             //              // scope
        int x { 0 }; // scope A     //              //
    }                //             // main()'s     //
                                    // scope        //
    {                //             //              //
        int x { 1 }; // scope B     //              //
    }                //             //              //
}                                   //              //
\end{lstlisting}

\noindent
In the above program, notice that \inlinecxx{x} is declared twice within \inlinecxx{main()}.
Typically, this would cause a duplicate symbol error.
In this case, however, the declarations occur within disjoint scopes, meaning the compiler is able to discern between them.
At any point in the program, the compiler recognizes symbols within the current scope as well as any enclosing scope.

\subsection{Global Scope}

\noindent
Global scope refers to the scope outside of any class, function, or namespace.
Variables declared within global scope are called \textbf{global variables} and can be accessed from anywhere in the program.
Global variables are stored in static memory ($\S$ \ref{NULLREF}) and their lifetime is the entire duration of the program.

\begin{lstlisting}[style=cxx]
#include <iostream>

int g { 42 };

void foo()
{
    std::cout << g << std::endl; // access global from foo()
}

int main()
{
    std::cout << g << std::endl; // access global from main()
    foo();
}
\end{lstlisting}

\subsection{Local Scope}

\noindent
Local scope refers to the region within a block of code defined by curly braces \inlinecxx{\{\}}.
This scope limits the variable's visibility and lifetime to that specific block:

\begin{lstlisting}[style=cxx]
#include <iostream>

void foo()
{
    int local { 10 };
    std::cout << local << std::endl;
}

int main()
{
    std::cout << local << std::endl; // error!
}
\end{lstlisting}

\subsection{Local Scope}
